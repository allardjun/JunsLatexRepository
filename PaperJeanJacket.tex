% This is the template Jun uses for papers
% jun.allard@uci.edu allardlab.com
\documentclass[onecolumn,11pt]{article}

% --------- required packages ---------
\usepackage{enumerate}
\usepackage{enumitem}
\usepackage{graphicx}
\usepackage[colorlinks,allcolors=black]{hyperref}
\usepackage[font=small]{caption}

% Units, define microMolar, picoNewton, micron, etc.
\usepackage{siunitx}
\sisetup{per-mode=reciprocal} % Change here to get slash or ^{-1}
\DeclareSIUnit{\million}{\text{million}}
\DeclareSIUnit{\Units}{\text{Units}}
\DeclareSIUnit{\cells}{\text{cells}}
\DeclareSIUnit\Molar{\mole\per\cubic\deci\metre}
\DeclareSIUnit\Molar{\textsc{m}}
\newcommand{\uM}{\si{\micro\Molar}}
\newcommand{\um}{\si{\micro\metre}}
\newcommand{\pN}{\si{\pico\newton}}
\newcommand{\s}{\si{\second}}
\newcommand{\nm}{\si{\nano\metre}}

\usepackage{times}  
\usepackage{titlesec}      

\usepackage{titling}
\usepackage{authblk} % must come after titling package loading

% --------- standard packages ---------
%\usepackage{amsfonts}
\usepackage{amsmath}      
\usepackage{amssymb} 
%\usepackage{array}  
%\usepackage{bm}
%\usepackage{color}
%\usepackage{chemarr}
%\usepackage{float} 
\usepackage[utf8]{inputenc}
\usepackage{lipsum}
%\usepackage{mathtools}
%\usepackage{multicol}
%\usepackage{multirow}
%\usepackage{pdfsync}
%\usepackage{tocloft}
\usepackage{verbatim} % This also provides the \comment{} environment
\renewcommand{\comment}[1]{{\textcolor{blue}{{#1}}}}

%\usepackage{wrapfig}


% --------- packages, less often used ---------
% It's good to keep these commented out until you need them, to reduce compilation time.
%\usepackage[export]{adjustbox}
%\usepackage{amsopn}
%\usepackage{cancel} % adds strikethrough
%\usepackage{CJK} % Chinese Japanese Korean
%\usepackage{dcolumn} % align columns in a table
%\usepackage{dsfont}
%\usepackage{epsfig}
%\usepackage{epstopdf}
%\usepackage{esint} % integrals
%\usepackage{framed}
%\usepackage{physics} %for \vqty, \dd, \dv
%\usepackage{rotating}
%\usepackage{setspace} % set space between lines
\usepackage{subcaption}

%\usepackage{tcolorbox}
%\usepackage[normalem]{ulem} % for underlining
% --------- for typesetting code blocks --------
%\usepackage{algorithmic} % for code 
%\usepackage{listings} % for code 
%\usepackage[numbered,framed,basicstyle]{matlab-prettifier}

%%%%%%%%%%%%%%%%%%%%%%%%%%%%%%%%%%%%%%%%%%%%%%%%%%%%%%%%%%%%
% page layout
\usepackage[margin=1in]{geometry}

% paragraph layout
\setlength{\parindent}{0pt} % paragraph indent - looks best zero
\setlength{\parskip}{0.8em plus 0.1em minus 0.1em} % paragraph spacing
\setlist[itemize]{itemsep=0.0em,topsep=0em}
\setlist[enumerate]{itemsep=0.0em,topsep=0em}
\titlespacing{\paragraph}{0em}{-0.8em}{1em} % for the /paragraph command

\clubpenalty10000
\widowpenalty10000

% \def\alphaoff{\alpha_{\mbox{\scriptsize off}}}
% \def\alphaon{\alpha_{\mbox{\scriptsize on}}}
% \def\cos2{\mbox{cos$^2$}}
% \def\sin2{\mbox{sin$^2$}}
% \def\tan2{\mbox{tan$^2$}}
% \def\aka{\emph{a.k.a. }}
% \def\boff{^b_{\mbox{\scriptsize off}}}
% \def\bon{^b_{\mbox{\scriptsize on}}}
% \def\bsub{\emph{B. subtilis}}
% \def\caul{\emph{C. crescentus}}
% \def\cosh{\mbox{cosh}\,}
% \def\cos{\mbox{cos}\,}
\def\deg{^\circ}
% \def\denovo{\emph{de novo}}
% \def\ds{\partial_s}
% \def\dt{\partial_t}
% \def\ecoli{\emph{E. coli}}
% \def\eg{\emph{e.g. }}
% \def\fcat{f_{cat}}
% \def\fgp{f_{gp}}
% \def\fgs{f_{gs}}
% \def\fpg{f_{pg}}
% \def\fps{f_{ps}}
% \def\fres{f_{res}}
% \def\fsg{f_{sg}}
% \def\fsp{f_{sp}}
% \def\fstall{f_{\mbox{\scriptsize stall}}}
% \def\ie{\emph{i.e. }}
% \def\invitro{\emph{in vitro}}
% \def\invitro{{in vitro}}
% \def\invivo{\emph{in vivo}}
% \def\invivo{{in vivo}}
% \def\kback{k_{back}}
% \def\kboff{k^b_{\mbox{\scriptsize off}}}
% \def\kbon{k^b_{\mbox{\scriptsize on}}}
% \def\khyd{k_{\mbox{\scriptsize hyd}}}
% \def\khyd{k_{hyd}}
% \def\knuc{k_{nuc}}
% \def\koff{k_{\mbox{\scriptsize off}}}
% \def\kon{k_{\mbox{\scriptsize on}}}
% \def\kpoff{k^p_{\mbox{\scriptsize off}}}
% \def\kpon{k^p_{\mbox{\scriptsize on}}}
% \def\min{\,\mbox{min}}
% \def\muM{\,\mu\mbox{M}}
% \def\mum{\,\mu\mbox{m}}
% \def\nm{\,\mbox{nm}}
% \def\pN{\,\mbox{pN}}
% \def\pclaspbg{p_{\mbox{\scriptsize CLASP-bg}}}
% \def\pclasp{p_{\mbox{\scriptsize CLASP}}}
% \def\pcross0{p_{\mbox{\scriptsize cross}}^0}
% \def\pcrossAA{p_{\mbox{\scriptsize cross}AA}}
% \def\pcrossAP{p_{\mbox{\scriptsize cross}AP}}
% \def\poff{^p_{\mbox{\scriptsize off}}}
% \def\poff{^p_{\mbox{\scriptsize off}}}
% \def\pon{^p_{\mbox{\scriptsize on}}}
% \def\pside{p_{\mbox{\scriptsize side}}}
% \def\rhod{\emph{R. sphaeroides}}
% \def\sech{\mbox{sech}\,}
% \def\sinh{\mbox{sinh}\,}
% \def\sin{\mbox{sin}\,}
% \def\s{\,\mbox{s}}
% \def\tan{\mbox{tan}\,}
% \def\therm{\emph{Thermotoga}}
% \def\vms{v^m_s}
% \def\vpg{v^p_g}
% \def\vps{v^p_s}
% \def\Ara{\emph{Arabidopsis}}
% \def\ara{\emph{Arabidopsis}}
% \def\vfree{v_{\mbox{\scriptsize free}}}
% \def\vpoly{v_{\mbox{\scriptsize poly}}}
% \def\Fstall{F_{\mbox{\scriptsize stall}}}
%Frequely used notations
\newcommand{\KD}{K\textsubscript{D}\xspace}
\newcommand{\koff}{$k$\textsubscript{off}\xspace}
\newcommand{\kon}{$k$\textsubscript{on}\xspace}
\newcommand{\ecf}{EC\textsubscript{50}\xspace}
\newcommand{\icf}{IC\textsubscript{50}\xspace}
% \newcommand{\approxt}{{$\sim$}}
% \newcommand{\Ag}{\text{Ag}}
% \newcommand{\Agsur}{\text{Ag}_{\text{sur}}}
% \newcommand{\Ab}{\text{Ab}}
% \newcommand{\D}[2]{\frac{d#1}{d#2}}
% \newcommand{\Cp}{C_{\text{p}}}
% \newcommand{\ts}{t_{\text{s}}}
% \newcommand{\konm}{k_{\text{on}}}
% \newcommand{\khatonm}{\hat{k}_{\text{on}}}
% \newcommand{\koffm}{k_{\text{off}}}
% \newcommand{\konbm}{k_{\text{on,b}}}
% \newcommand{\ind}{\mathbbm{1}}
% \newcommand{\E}{\mathbb{E}}
% \newcommand{\avg}[1]{\E[#1]}
% \newcommand{\nocc}{\mathcal{N}}
% \newcommand{\Nag}{N_{\text{Ag}}}
% \newcommand{\Rp}{R_{\text{p}}}
% \newcommand{\vtheta}{\boldsymbol{\theta}}
% \newcommand{\Na}{N_{\text{A}}}
% \newcommand{\Rs}{R_{\text{s}}}
% \newcommand{\ubar}[1]{\underaccent{\bar}{#1}}
% \newcommand{\Nsims}{N_{\text{sims}}}
% \newcommand{\reach}{\varepsilon}
% \newcommand{\reachsur}{\reach_{\text{sur}}}
% \newcommand{\reachphys}{\reach_{\text{phys}}}



% Bibliography setup
\usepackage[numbers,square,sort&compress]{natbib}
\bibliographystyle{unsrtnat}
%\bibliographystyle{biophysj}

\title{JeanJacket main result statement as the title}
\author[a]{Jun}
\affil[a]{University of California Irvine}


%%%%%%%%%%%%%%%%%%%%%%%%%%%%%%%%%%%%%%%%%%%%%%%%%%%%%%%
\date{} % leave blank for no date
\begin{document}

\maketitle
%%%%%%%%%%%%%%%%%%%%%%%%%%%%%%%%%%%%%%%%%%%%%%%%%%%%%%%

\begin{abstract}
Nothing here yet
\end{abstract}

%%%%%%%%%%%%%%%%%%%%%%%%%%%%%%%%%%%%%%%%%%%%%%%%%%%%%%%
% \textbf{Key words:} 

% \textbf{Pre-print server:} biorxiv

% %\textbf{Open access:} 

% \clearpage

% \renewcommand{\abstractname}{Abstract:}
% \begin{abstract}

% \end{abstract}

% \renewcommand{\abstractname}{Significance Statement:}
% \begin{abstract}
% \end{abstract}

%\renewcommand{\abstractname}{Graphical abstract:}
%\begin{abstract}
%\begin{figure}[!htb]\vspace{-0.8cm}
%	\begin{center}
%	\hspace*{-0.25in}
%	\includegraphics[width=12cm]{figures/figGraphicalAbstract.pdf}\vspace{-0.5cm}
%    \captionsetup{labelformat=empty} 	\end{center}
%\end{figure} \setcounter{figure}{0}   
%\end{abstract}

%\clearpage
%\linenumbers
%%%%%%%%%%%%%%%%%%%%%%%%%%%%%%%%%%%%%%%%%%%%%%%%%%%%%%%



\section*{Introduction}

\paragraph{Background/major question} \lipsum[2]
\paragraph{Specific question, puzzle or hypothesis}\lipsum[2]
\paragraph{Here we}\lipsum[2]
\paragraph{Methods, tools, expertise, skills}\lipsum[2]
\paragraph{Expected results}\lipsum[2]
\paragraph{Outcomes and implications}\lipsum[2]


%%%%%%%%%%%%%%%%%%%%%%%%%%%%%%%%
\begin{figure}[h!]
\centering
\includegraphics[width=4.5in]{figures/figJeanJacket.pdf}
\caption{\label{fig:JeanJacketSchematic}An awesome model of a puzzling phenomenon}
\end{figure}
%%%%%%%%%%%%%%%%%%%%%%%%%%%%%%%%

\subsection*{Candidate titles}
 
\begin{itemize}
\item Predictive dynamic modeling connects timescales of day-to-day gene expression and decade-scale disease progression of Alzheimer's Disease using $\mathbb{C}^\star$ algebras
\item Ito lemma and Boltzmann statistics provide a cost-effective treatment for cardiovascular disease
\item Hybrid Metropolis-Gillespie algorithm reveals a novel design for enhanced CAR-T therapy by dynamnic recruitment of non-effector molecular crowders to the T cell membranes
\item Comparative study of noncanonical protrusions and what they optimize
\item Accelerated tendon healing by morphological alterations to tenocyte projections
\item Universal model that achieves both zero bias and zero variance between training sets
\end{itemize}

\subsection*{Literature}
Primarily modeling: \citet{Miller.2018}, \citet{Wilhelm.2003}. 
Some examples of textual citation where we talk about the work of \citet{Miller.2018}, versus parenthetical citation where we mention a fact \cite{Miller.2018}. 


%%%%%%%%%%%%%%%%%%%%%%%%%%%%%%%%%%%%%%%%%%%%%%%%%%%%%%%%%%%%%%%%%%%%%%%%%%%%%%%%%%%
\clearpage % Remove before submission. 
\section*{Results} 


\subsection*{An active sentence about this result, which roughly corresponds to Figure~\ref{fig:majorfig}} 


Cool latex tricks for physical units: Test dimensions. 
Dimensional analysis. Pull the other cell with \SI{10}{\pico\newton\per\second}. Deviatoric stress \SI{7e3}{\pascal}. Inside a math environment $R=\SI{120}{\micro\meter}$. 
Something in inverse microMolar $10\uM^{-1}$, and force in $10\pN$ and distance in either $10\um$ or $10\nm$

Cool trick for figure referencing: We should be able to reference Figure~\ref{subfig:heatmap} and Figure~\ref{subfig:lattice} separately.

We can also refer to supplemental figures like Figure~\ref{fig:DetailedJeanJacketSchematic}.
Note about figure size: for many journals, the best and/or required sizes are either 8.5cm or 17cm.
Try to make the size in Illustrator and latex (in \verb|includegraphics|) match. 
However, slightly-shrunken 165mm looks a bit better in latex.

\begin{figure}[!ht]
        \centering
        \begin{subcaptiongroup}
                \includegraphics[width=165mm]{figures/figJeanJacket.pdf}
                \phantomcaption\label{subfig:heatmap}
                \phantomcaption\label{subfig:lattice}
                \end{subcaptiongroup}
        \captionsetup{subrefformat=parens}
        \caption{\label{fig:majorfig}\textbf{Interpretation sentence.}
                \subref{subfig:heatmap} A good caption for the heatmap. 
                \subref{subfig:lattice} A good caption for the lattice. 
                }
\end{figure}

\lipsum[2]  


%%%%%%%%%%%%%%%%%%%%%%%%%%%%%%%%%%%%%%%%%%%%%%%%%%%%%%%%%%%%%%%%%%%%%%%%%%%%%%%%%%%
\clearpage % Remove before submission. 
\subsection*{Active sentence about the 2nd logical unit of results, roughly corresponding to Figure~\ref{fig:2ndresult}}

\lipsum[2-4]  
\begin{figure}[!ht]
        \centering
        \includegraphics[width=165mm]{figures/figJeanJacket.pdf}
        \caption{\label{fig:2ndresult}\textbf{Interpretation sentence.}
               The second logical unit of results, which will naturally lead to the third, and so on.
                }
\end{figure}
\lipsum[5-6]  


% Bibliography
\bibliography{jeanjacket_jun.bib}
% You can include multiple .bib files, so leave "_jun" and make a "_yourname.bib" file.

%%%%%%%%%%%%%%%%%%%%%%%%%%%%%%%%%%%%%%%%%%%%%%%%%%%%%%%%%%%%%%%%%%%%%%%%%%%%%

% Supplemental
\clearpage
\setcounter{table}{0}
        \renewcommand{\thetable}{S\arabic{table}}%
\setcounter{figure}{0}
        \renewcommand{\thefigure}{S\arabic{figure}}%
\renewcommand{\listfigurename}{List of Supporting Figures}
\renewcommand{\contentsname}{List of Supporting Text}

\section{Supplemental material}

%%%%%%%%%%%%%%%%%%%%%%%%%%%%%%%
\begin{figure}[h!t]
\centering
\includegraphics[width=4.5in]{figures/figJeanJacket.pdf}
\caption{This is a supplemental figure}
\label{fig:DetailedJeanJacketSchematic}
\end{figure}
%%%%%%%%%%%%%%%%%%%%%%%%%%%%%%%

\end{document}
